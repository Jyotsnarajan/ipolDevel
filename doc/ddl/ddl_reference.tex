\section{Introduction}
The Demo Description Lines (DDL) is an abtract syntax which allow to
define an IPOL demo. The description is 
written in JSON (JavaScript Object Notation) format. This language is
evolving to allow a maximum number of demos to be described without the
need to manually write HTML or Python code 
for a given demo. Each main key in the description file is described
in the following sections:
\begin{itemize}
  \item \textit{general}: general options (required);
  \item \textit{build}: download and compile the source code (required);
  \item \textit{inputs}: description of the different inputs  (required);
  \item \textit{params}: description of the parameters, for the param page (recommended);
  \item \textit{run}: commands to run the demo (required);
  \item \textit{config}: configuration, saving demo information (optional);
  \item \textit{archive}: description of files  what is saved in archive when inputs are uploaded (optional);
  \item \textit{results}: displaying the result page (required).
\end{itemize}

\vspace{1em}

JSON data for IPOL should be strictly valid. Most common mistakes:
\begin{itemize}
  \item ASCII, no letters with diacritics or accents like 'é' or '$\sigma$', use HTML entities for special characters (ex: \&eacute; \&sigma;);
  \item commas, requested separator between items in list or dictionary, forbidden at start or end of a collection.
\end{itemize}
The IPOL control panel provide a JSON editor with a simple validator. Most of the syntax errors are detected in real time
and reported by a graphic hint. The non ASCII characters are not detected by the client. 
Before to be applied, the JSON data are validated by the server, and may be not accepted.

%-------------------------------------------------------------------------------
\section{The \emph{general} section}
The general section describes global information about the demo.
It is a set of  (key,value) pairs, described in the following table.
Many keys are derived from the static variables of the previous 'app' Python class.\\
% 
\begin{longtable}{|>{\bf}L{\dimexpr 0.28\linewidth}|L{\dimexpr 
0.57\linewidth}|c|}
\hline
 \centering {key}     & \centering {\bf description} & {\bf req} 
\tabularnewline \hline \hline
 demo\_title         & demo title & yes\\ \hline
 input\_description  & description at the top of the input selection page, 
                      contains html as a single string or as an array of string
                      that will be concatenated and separated with spaces.
                     & yes \\ \hline
 param\_description  & description at the top of the para\-meters page,
                      contains html as a single string or as an array of string
                      that will be concatenated and separated with spaces.
                      & yes
                      \\ \hline
 xlink\_article     & defines the link to the article webpage & yes  \\ \hline
 crop\_maxsize      & limit allowed crop size in both width and height, the string
                      can contain a JavaScript expression which will be evaluated & no \\ \hline
 requirements 	    & specify the particular requirements needed for the execution of the demo separated by commas. e.g. Matlab. & no \\ \hline
 timeout 	    & specify the time in seconds to execute the algorithm. If the execution takes longer than the specified time 
		      the system kills the process.& no \\ \hline
\caption{Keys for the 'general' section ({\em req} means required).}
\end{longtable}

%-------------------------------------------------------------------------------
\section{The \emph{build} section}

The build section contains an array of build descriptions in the form
''[ \{...\}, \{...\}, ...] ''. For each description, an archive containing the 
source code is downloaded and compiled using either 'make','cmake' or 'script' features.
It has the following information:

\subsection{\emph{make} type}

\begin{longtable}{|>{\bf}L{\dimexpr 0.25\linewidth}|L{\dimexpr 0.6\linewidth}|c|}
\hline
\centering {key}     & \centering {\bf description} & {\bf req} \tabularnewline 
\hline \hline
 build\_type    & make & yes \\ \hline
 url        & full url link to download the demo source code & yes \\ \hline
 srcdir     & subdirectory from the extracted archive where the source code is 
            located & yes \\ \hline
 prepare\_make & shell command used to fix compilation errors of the source code,
                since the source code is steady & no  \\ \hline
 binaries   & list of binaries and their associated paths relative to the source 
            code path. In case of 'make' build type, the first path is the compilation
            path, where the make command is called, the second path is the binary
            path. You can run make without a binary name by setting the second
            path to a directory (it should end with '/' character), then all files 
            from this directory will be copied
            to the demo binary path. If the second path is a directory, give 
            a third value corresponding to one of the files to be copied, so that
            the system can check the file timestamp to decide if rebuild is needed.
            & yes \\ \hline
 flags      & make command compilation flags & yes \\ \hline
 scripts    & list of scripts and their associated paths relative to the source 
            code path & no  \\ \hline
 post\_build & shell command to run after the build & no \\ \hline
\caption{Keys for the 'make' type.}
\end{longtable}

\paragraph{Example}:\\
\begin{lstlisting}[language=json,firstnumber=1]
"build": [ 
  {
    "build_type"    : "make",
    "url"           : "http://www.ipol.im/../phs_3.tar.gz", 
    "srcdir"        : "phs_3",
    "binaries"      : [ [".","horn_schunck_pyramidal"] ],
    "flags"         : "-j4"
  {
    "build_type"    : "make",
    "url"           : "http://www.ipol.im/../imscript_dec2011.tar.gz", 
    "srcdir"        : "imscript",
    "binaries"      : [ [".","bin/", "plambda"] ],
    "flags"          : "-j CFLAGS=-O3 IIOFLAGS='-lpng -lm'"
  }
]
\end{lstlisting}

\subsection{\emph{cmake} type}

\begin{longtable}{|>{\bf}L{\dimexpr 0.25\linewidth}|L{\dimexpr 0.6\linewidth}|c|}
\hline
\centering {key}     & \centering {\bf description} & {\bf req} \tabularnewline 
\hline \hline
 build\_type  & cmake & yes \\ \hline
 url          & same as make type & yes \\ \hline
 srcdir       & same as make type & yes \\ \hline
 prepare\_cmake & shell command used to fix compilation errors of the source code,
                since the source code is steady & no  \\ \hline
 cmake\_flags  & cmake flags for configuration ('Release' build type is 
                automatically set) & no  \\ \hline
 binaries     & same as make type & yes \\ \hline
 flags        & same as make type & yes \\ \hline
 scripts      & same as make type & no  \\ \hline
 post\_build  & same as make type & no \\ \hline
\caption{Keys for the 'cmake' type.}
\end{longtable}

\paragraph{Example}:\\
\begin{lstlisting}[language=json,firstnumber=1]
"build": [ { 
  "build_type" : "cmake",
  "url"        : "http://www.ipol.im/xxx/ldm_q1p.zip", 
  "srcdir"     : ".",
  "binaries"   : [ [".","lens_distortion_correction"] ],
  "flags"      : "OMP=1 -j4" } ]
\end{lstlisting}

\subsection{\emph{script} type}

\begin{longtable}{|>{\bf}L{\dimexpr 0.25\linewidth}|L{\dimexpr 0.6\linewidth}|c|}
\hline
\centering {key}     & \centering {\bf description} & {\bf req} \tabularnewline 
\hline \hline
 build\_type  & script & yes \\ \hline
 url          & same as make type & yes \\ \hline
 srcdir       & same as make type & yes \\ \hline
 scripts      & same as make type & no  \\ \hline
\caption{Keys for the 'script' type.}
\end{longtable}

\paragraph{Example}:\\
\begin{lstlisting}[language=json,firstnumber=1]
"build": [ { 
    "build_type": "script",
    "url": "http://151.80.24.28:8080/DemoSource/1000003/demo_scripts.tgz",
    "srcdir": "demo_scripts",
    "scripts": [ [ ".", "write_line_parameters.py" ] ]
  }
\end{lstlisting}


%-------------------------------------------------------------------------------
\section{The \emph{inputs} section}
The inputs section describes the set of input data of the algorithm (images, 
videos, flows, etc..).

\subsection{\emph{image} type}

\begin{longtable}{|>{\bf}L{\dimexpr 0.26\linewidth}|L{\dimexpr 0.59\linewidth}|c|}
\hline
\centering {key}     & \centering {\bf description} & {\bf req} \tabularnewline 
\hline \hline
 type         & image & yes \\ \hline
 description  & Short name or description of the input, it will used to display 
the input as the label inside the 'gallery' html display & yes \\ \hline
 max\_pixels   &  Sets the maximal number of pixels of the input image, 
bigger images will be downsized, if 0 no resizing is done. 
Input can be a number or a simple arithmetic expression as a string that will be evaluated (ex: "1000*1000" = 1 Mpx).  & .. \\ \hline
 max\_weight   & Maximum weight (in bytes) of an input file, prevents uploading 
bigger files.
Input can be a number or a simple arithmetic expression as a string that will be evaluated (ex: "100*1024*1024"= 100 Mb). & .. \\ \hline
 dtype        & Expected data type for image to process:
\vspace{-1em}
\begin{itemize}
  \setlength\itemsep{-0.5em}
  \item \textit{1x8i}: gray, unsigned integer 8 bits;
  \item \textit{3x8i}: color, RGB unsigned integer 8 bits;
  \item \textit{1x16i}: gray, unsigned integer 16 bits;
  \item \textit{3x16i}: color, RGB unsigned integer 16 bits.
\end{itemize} & .. \\ \hline
 ext          & input image expected extension (ie. file format) & .. \\ \hline
forbid\_preprocess         & Forbids any pre-processing of the input data by IPOL system. 
Submitted image is kept as-is.
Used by algorithms like noise-estimation or modification detection, where re-sampling will affect results. 
If a processing is needed, according to the expected properties upper, user will be informed by a message.
& no \\ \hline
\caption{Keys for the 'image' type.}
\end{longtable}

%-------------------------------------------------------------------------------
\section{The \emph{params} section}
The params section describes the set of parameters needed by a demo, their 
constraints and their visual appearance. It is defined as an array of sets, 
where each set contains (key,value) pairs.


\subsection{ \emph{range} type}

The values of the range type are stored as numbers, so no double quotes are 
required around the values. The default value is stored with the 'values' field.

\begin{longtable}{|>{\bf}L{\dimexpr 0.15\linewidth}|L{\dimexpr 
0.7\linewidth}|c|}
\hline
 \centering {key}     & \centering {\bf description} & {\bf req} 
\tabularnewline \hline \hline
 type  & range       & yes \\ \hline
 visible  & evaluated Javascript boolean expression. It hides the parameter if
            evaluated to false (if undefined the parameter is visible)& no \\ \hline
 id     & parameter name in lowercase letters  & yes \\ \hline
 label  & name and/or description of the parameter, appears on the left side. & yes
                      \\ \hline
 comments & description of the parameter, appears on the right side. & no
                      \\ \hline
 values & set min,max,step and default values using the following key/value 
scheme \{ 'min':val, 'max':val, 'step':val, 'default':val \} & yes
                      \\ \hline
\caption{Keys for the 'range' type.}
\end{longtable}


\subsection{ \emph{selection\_collapsed} type}

The values of the selection are stored as strings, so we map each label with a 
string value. The default value is stored in a separate field.

\begin{longtable}{|>{\bf}L{\dimexpr 0.25\linewidth}|L{\dimexpr 
0.6\linewidth}|c|}
\hline
 \centering {key}     & \centering {\bf description} & {\bf req} 
\tabularnewline \hline \hline
 type  & selection\_collapsed    & yes \\ \hline
 visible  & Evaluated Javascript boolean expression which hides the parameter if
            evaluated to false (if undefined the parameter is visible)& no \\ \hline
 id     & parameter name in lowercase letters & yes \\ \hline
 label  & name and/or description of the parameter, appears on the left side. & yes
                      \\ \hline
 comments & description of the parameter, appears on the right side. & no
                      \\ \hline
 values & set of (key,value) pairs, where the key is the displayed text and the 
value a string representing the corresponding value, for example \{ 
'black':'0', '1\%':'0.01' \} & yes
                      \\ \hline
 default\_value & defines the default value for this parameter, should be one 
the values defined in 'values'. & yes \\ \hline
\caption{Keys for the 'selection\_collapsed' type.}
\end{longtable}

\subsection{ \emph{selection\_radio} type}

The values of the selection are stored as strings, so we map each label with a 
string value. The default value is stored in a separate field.
This type is similar to the selection collapsed type, but display the different
options using radio buttons.

\begin{longtable}{|>{\bf}L{\dimexpr 0.25\linewidth}|L{\dimexpr 
0.6\linewidth}|c|}
\hline
 \centering {key}     & \centering {\bf description} & {\bf req} 
\tabularnewline \hline \hline
 type     & selection\_radio    & yes \\ \hline
 visible  & Evaluated Javascript boolean expression which hides the parameter if
            evaluated to false (if undefined the parameter is visible)& no \\ \hline
 id       & parameter name in lowercase letters & yes \\ \hline
 label  & name and/or description of the parameter, appears on the left side. & yes
                      \\ \hline
 comments & description of the parameter, appears on the right side. & no
                      \\ \hline
 vertical & boolean, if true use vertical display, otherwise use horizontal
            display (default=false) & no \\ \hline
 values   & set of (key,value) pairs, where the key is the displayed text and the 
value a string representing the corresponding value, for example \{ 
'black':'0', '1\%':'0.01' \} & yes
                      \\ \hline
 default\_value & defines the default value for this parameter, should be one 
the values defined in 'values'. & yes \\ \hline
\caption{Keys for the 'selection\_radio' type.}
\end{longtable}

\subsection{ \emph{label} type}

The label  can be used as a title to separate groups of parameters.

\begin{longtable}{|>{\bf}L{\dimexpr 0.27\linewidth}|L{\dimexpr 
0.58\linewidth}|c|}
\hline
 \centering {key}     & \centering {\bf description} & {\bf req} 
\tabularnewline \hline \hline
 type  & label       & yes \\ \hline
 visible  & Evaluated Javascript boolean expression which hides the parameter if
            evaluated to false (if undefined the parameter is visible)& no \\ \hline
 label & html text to display, as a single string or as an array of strings& yes
                      \\ \hline
\caption{Common keys for the 'label' type.}
\end{longtable}


\subsection{ \emph{checkbox} type}

\begin{longtable}{|>{\bf}L{\dimexpr 0.27\linewidth}|L{\dimexpr 
0.58\linewidth}|c|}
\hline
 \centering {key}     & \centering {\bf description} & {\bf req} 
\tabularnewline \hline \hline
 type  & checkbox  & yes \\ \hline
 visible  & Evaluated Javascript boolean expression which hides the parameter if
            evaluated to false (if undefined the parameter is visible)& no \\ \hline
 label  & name and/or description of the parameter, appears on the left side. & yes
                      \\ \hline
 comments & description of the parameter, appears on the right side. & no
                      \\ \hline
 id         & parameter name in lowercase letters  & yes \\ \hline
 default\_value & boolean: True for checked & \\ \hline
\caption{Common keys for the 'checkbox' type.}
\end{longtable}

\subsection{ \emph{checkboxes} type}

\begin{longtable}{|>{\bf}L{\dimexpr 0.27\linewidth}|L{\dimexpr 
0.58\linewidth}|c|}
\hline
 \centering {key}     & \centering {\bf description} & {\bf req} 
\tabularnewline \hline \hline
 type  & checkboxes   & yes \\ \hline
 visible  & Evaluated Javascript boolean expression which hides the parameter if
            evaluated to false (if undefined the parameter is visible)& no \\ \hline
 label  & name and/or description of the parameter, appears on the left side. & yes
                      \\ \hline
 comments & description of the parameter, appears on the right side. & no
                      \\ \hline
 id         & parameter name in lowercase letters  & yes \\ \hline
 values     & list of dictionaries: [ {key:value, ...}, ... ]
            where key is the checkbox id and value is the text associated
            & yes \\ \hline
 default     & list of checkboxes that are checked by default  & yes \\ \hline
\caption{Common keys for the 'checkboxes' type.}
\end{longtable}


\subsection{ \emph{numeric} type}

This param allows just numeric input
\begin{longtable}{|>{\bf}L{\dimexpr 0.27\linewidth}|L{\dimexpr 
0.58\linewidth}|c|}
\hline
 \centering {key}     & \centering {\bf description} & {\bf req} 
\tabularnewline \hline \hline
 type       & numeric   & yes \\ \hline
 visible    & Evaluated Javascript boolean expression which hides the parameter
              if evaluated to false (if undefined the parameter is visible) & no \\ \hline
 label      & name and/or description of the parameter, appears on the left side. & yes \\ \hline
 comments   & description of the parameter, appears on the right side. & no  \\ \hline
 id         & parameter name in lowercase letters & yes \\ \hline
 values     & set min,max,step and default values using the following key/value scheme \{ 'min':val, 'max':val, 'default':val \} & no \\ \hline
\caption{Common keys for the 'checkboxes' type.}
\end{longtable}

\subsection{ \emph{text} type}

This param allows numeric and text input
\begin{longtable}{|>{\bf}L{\dimexpr 0.27\linewidth}|L{\dimexpr 
0.58\linewidth}|c|}
\hline
 \centering {key}     & \centering {\bf description} & {\bf req} 
\tabularnewline \hline \hline
 type       & text   & yes \\ \hline
 visible    & Evaluated Javascript boolean expression which hides the parameter
              if evaluated to false (if undefined the parameter is visible) & no \\ \hline
 label      & name and/or description of the parameter, appears on the left side. & yes \\ \hline
 id         & parameter name in lowercase letters & yes \\ \hline
 values     & set maxlength and default values using the following key/value scheme \{ 'maxlength':val, 'default':val \} & no \\ \hline
\caption{Common keys for the 'checkboxes' type.}
\end{longtable}

%-------------------------------------------------------------------------------
\section{The \emph{params\_layout} section}

The default layout of parameters is to stack them together within an 'HTML field'
entitle 'Parameters:'. However, if a demo needs to group together some parameters,
it can use a specific params\_layout section that contains an array of
sets of parameters, where each set is defined by an array containing the group 
title as the first element and the list of parameters ids as the second element,
where the parameters ids are their position in the params section (starting at 0).

\paragraph{Examples}:\\
\begin{lstlisting}[language=json,firstnumber=1]
  "params_layout": [
        [ "General parameters:",          [ 0,1]  ],
        [ "Sampled Gaussian kernel",      [2]     ],
        [ "Lindeberg's smoothing method", [3]     ]
    ],
\end{lstlisting}

%-------------------------------------------------------------------------------
\section{The \emph{run} section}

The 'run' section contains an array of elements, where each element can be 
a string or another array of strings:
\begin{itemize}
\item if the element is a string starting with '\#', it is considered as a comment.
\item if the element is a string, it can be a command to run from the system (shell, 
python script or demo binary) or a python code to evaluate if the string
starts with "python:",
\item if the element is an array of strings, the first string is a python
condition, if this condition is fulfilled, the remaining commands will be
executed.
\end{itemize}
In the command line, you can use \$param\_id to evaluate a 
python expression that can contain the parameter ids,
and '\textgreater output\_file' (without space between '\textgreater' and the 
filename) to redirect the standard output to a given file, save in the current 
working directory. 
If you also want to redirect the standard errors to the same file, use can add 
'2\textgreater\&1' as another argument. However, pipelines are not allowed 
('|' character).
Python scripts from the PythonTools directory can be used in commands: currently
'count\_lines.py' and 'image\_histogram.py'.
If you want to access specific demo data copied in the binary directory, use must
use the variable 'demodata', for example "cp \$\{demodata\}/\-pattern\-\_noise.pgm .".


\paragraph{Examples}:\\
\begin{lstlisting}[language=json,firstnumber=1]
"run": [
  "#--- comment ---"
  "nlmeans  input_0.sel.png $sigma input_1.png output_1.png",
  "img_diff input_0.sel.png input_1.png $sigma output_2.png",
  "img_mse  input_0.sel.png output_1.png >stdout.txt 2>&1"
]
\end{lstlisting}

In this example, the last command adds a new parameters using self.algo\_params,
this parameter will be saved in the config files. It is also possible to add 
or change info and meta sections using self.algo\_info and self.algo\_meta.
\begin{lstlisting}[language=json,firstnumber=1]
  "run": [ 
    "mosaic     -p $pattern input_0.sel.png mosaiced.png ",
    "python:sizeX=x1-x0;sizeY=y1-y0",
    "python:zoom_factor=max(1, int(math.ceil(200.0/min(sizeX, sizeX))))",
    "python:sizeX=sizeX*zoom_factor;sizeY=sizeY*zoom_factor",
    "convert -filter point -resize ${sizeX}x${sizeY} input_0.sel.png    input_0.sel_zoom.png",
    "python:self.algo_params['zoom_factor']=zoom_factor"
  ]
\end{lstlisting}

A more complex example taken from demo 70:
\begin{lstlisting}[language=json,firstnumber=1]
  "run": [ 
      "convert input_0.sel.png inputNG.pgm",
      [ "threshold_type=='manual'", 
          "pgm2freeman -min_size $min_c -image inputNG.pgm -outputSDPAll -maxThreshold $tmax -minThreshold $tmin >inputPolygon.txt 2>algoLog.txt"
      ],
      [ "threshold_type=='auto'",   
          "pgm2freeman -min_size $min_c -image inputNG.pgm -outputSDPAll >inputPolygon.txt 2>algoLog.txt",
          "python:fInfo = open(self.work_dir+'algoLog.txt', 'r')",
          "python:lines = fInfo.readlines()",
          "python:line_cases = lines[0].replace(')', ' ').split();",
          "python:tmax = int(line_cases[17])"
      ],
      "python:contoursList = open (self.work_dir+'inputPolygon1.txt', 'w')",
      ...
  ]
\end{lstlisting}
 

%-------------------------------------------------------------------------------
\section{The \emph{config} section}

The config section is optional, it allows creating new (key,value) pairs in the 
configuration file based on text files obtained during the execution. This 
new information can then be used in the archive section.

\begin{longtable}{|>{\bf}L{\dimexpr 0.25\linewidth}|L{\dimexpr 0.6\linewidth}|c|}
\hline
\centering {key}     & \centering {\bf description} & {\bf req} \tabularnewline 
\hline \hline
 info\_from\_file    & it contains a list of (key,value) pairs where the key is
            the new information id to create (in lowercase letters) and the
            value is the corresponding text file that contains its contents & no \\ \hline
\caption{keys for the 'config' section.}
\end{longtable}

\paragraph{Example}:\\
\begin{lstlisting}[language=json,firstnumber=1]
"config":
  {
    "info_from_file": {  "homography_1" : "output_0.txt",
                          "homography_2" : "output_1.txt"
                      }
  }
\end{lstlisting}

%-------------------------------------------------------------------------------
\section{The \emph{archive} section}


\begin{longtable}{|>{\bf}L{\dimexpr 0.28\linewidth}|L{\dimexpr 0.6\linewidth}|c|}
\hline
\centering {key}     & \centering {\bf description} & {\bf req} \tabularnewline 
\hline \hline
 files    & (key,value) pairs where key is the file to archive and value is 
            the associated text information & no \\ \hline
 compressed\_\-files   & compressed files to add in the same format 
as files & no \\ \hline
 params  & list of parameters to archive & no \\ \hline
 info    & info variables to archive in the form of pairs variable:label & no \\ \hline
 enable\_reconstruct & Enable a button to reconstruct an experiment stored in the archive (default value: false) & no \\ \hline
 archive\_always     & The archive stores all the experiments executed included the default blobs (default value: false) & no \\ \hline
\caption{Keys for the 'text\_file' type.}
\end{longtable}


\paragraph{Example}:\\
\begin{lstlisting}[language=json,firstnumber=1]
"archive":
  {
    "enable_reconstruct": true,
    "archive_always":true,
    "files" : 
      { "input_0.png"                 : "input image",
        "primitives.txt"              : "Primitives"  },
    "params" :  
      [ "high_threshold_canny", 
        "initial_distortion_parameter", 
        "angle_point_orientation_max_difference" ],
    "info"   : { "run_time": "run time" }
  }
\end{lstlisting}

%-------------------------------------------------------------------------------
\section{The \emph{results} section}


The results section also contains an array of sets, where each set contains 
(key/value) pairs describing one type of output from the algorithm. There are 
displayed sequentially one below the other, a part from warnings that are 
displayed at the top of the page. The currently available types are described 
below. At the top of the results page, the processing time is displayed and the 
user is proposed to run the demo again with different input or parameters.


%------ gallery  ------
\subsection{ \emph{gallery} type}

The gallery type uses the gallery class to display images. Its parameters are:

\begin{longtable}{|>{\bf}L{\dimexpr 0.15\linewidth}|L{\dimexpr 0.7\linewidth}|c|}
\hline
 \centering {key}     & \centering {\bf description} & {\bf req} 
\tabularnewline \hline \hline
 type       & gallery  & yes \\ \hline
 visible    & Evaluated Javascript boolean expression 
              which hides the result if
              evaluated to false (if undefined the result is visible)& no \\ \hline
 label      & html label for the gallery, can be either a single string or 
             a list of string that will be concatenated with whitespace character
              in between. & no \\ \hline
 contents   & inside has Javascript objects with: img, repeat and visible parameters.
 				The key for every object can be an expression to be evaluated in case of repeat
 				functionallity.
 			    img wil have either a string with a filename or an array of strings with filenames.
 			    visible is the same a visible field for the gallery but for each item in the gallery.
 			    repeat is a range expression (evaluated in javascript): will create a loop in the 
 			    form idx=0..range-1.
            & yes \\ \hline
 options    & You can specify options for the ImageGallery object. Currently
              'minwidth' and 'minheight' (in pixels) options are available to force
              miminal size of image display. Default values are 300 for both.& no \\ \hline
\caption{Keys for the 'gallery' type.}
\end{longtable}

\paragraph{Examples}:\\
Advaced example, mixing repeat, visible, using an array of filenames.
\begin{lstlisting}[language=json,firstnumber=1]
{
  	"type": "gallery",
            "visible": "1==1",
            "contents": {
                "Input 0": {
                    "img":  ["denoised_diff_0.png", "denoised_noiseCurves_L_0.png"],
                    "visible": "1==1"
                },
                "'Scale'+idx": {
                    "img":  [
                        "'diff_'+idx+'.png'", 
                        "'noiseC_L_'+idx+'.png'", 
                        "'noiseC_H_'+idx+'.png'"
                    ],
                    "visible": "1==1",
                    "repeat": "params.nbscales"
                }
            },
            "label": [
                "<p>New Gallery type.</p>"
            ]
        }
\end{lstlisting}


%------ gallery_video  ------
\subsection{ \emph{gallery\_video} type}

The gallery\_video type uses the gallery class to display images. Its parameters are:

\begin{longtable}{|>{\bf}L{\dimexpr 0.15\linewidth}|L{\dimexpr 0.7\linewidth}|c|}
\hline
 \centering {key}     & \centering {\bf description} & {\bf req} 
\tabularnewline \hline \hline
 type       & gallery  & yes \\ \hline
 visible    & Evaluated Javascript boolean expression 
              which hides the result if
              evaluated to false (if undefined the result is visible)& no \\ \hline
 label      & html label for the gallery, can be either a single string or 
             a list of string that will be concatenated with whitespace character
              in between. & no \\ \hline
 contents   & inside has javascript objects with: video, repeat and visible parameters.
 				The key for every object can be an expression to be evaluated in case of the use of 
 				repeat functionallity.
 			    video wil have either a string with a filename or an array of strings with filenames.
 			    visible is the same a visible field for the gallery but for each item in the gallery.
 			    repeat is a range expression (evaluated in javascript): will create a loop in the 
 			    form idx=0..range-1.
            & yes \\ \hline
 options    & You can specify options for the VideoGallery object. Currently
              'minwidth' and 'minheight' (in pixels) options are available to force
              miminal size of video display. Default values are 300 for both.& no \\ \hline
\caption{Keys for the 'gallery\_new' type.}
\end{longtable}

\paragraph{Examples}:\\
Advaced example, mixing repeat, visible, using an array of filenames.
\begin{lstlisting}[language=json,firstnumber=1]
{
            "type": "video_gallery",
            "label": "<b>Video gallery</b>",
            "display": "grid",
            "visible": "1==1",
            "contents": {
                "Input_0": {
                    "video":  "'input_0.mp4'",
                    "visible": "1==1"
                },
                "'Scale_'+idx": {
                    "video":  "'scaled_'+idx+'.mp4'",
                    "repeat": "4"
                }
            }
        }
\end{lstlisting}



%------ file_download  ------
\subsection{ \emph{file\_download} type}

\begin{longtable}{|>{\bf}L{\dimexpr 0.15\linewidth}|L{\dimexpr 0.7\linewidth}|c|}
\hline
\centering {key}     & \centering {\bf description} & {\bf req} \tabularnewline 
\hline \hline
 type      & file\_download  & yes \\ \hline
 visible   & Evaluated Javascript boolean expression which hides the result if
             evaluated to false (if undefined the result is visible)& no \\ \hline
 repeat    & range expression (evaluated in Javascript):
              will create a loop in the form idx=0..range-1 & no \\ \hline
 label     & html title associated to the file download. In case of repeat, 
            evaluated as an expression with idx variable, otherwise, can be evaluated
            if it starts with single quote.& yes \\ \hline
 contents  & either a single string of the filename to download, or a list
              of label:filename pairs for files to download. In case of repeat, 
            evaluated as an expression with idx variable. & yes \\ \hline
\caption{Keys for the 'file\_download' type.}
\end{longtable}

\paragraph{Examples}:\\
\begin{lstlisting}[language=json,firstnumber=1]
  { "type"     : "file_download", 
    "label"    : "Download Hough result",
    "contents" : "output_hough.png" },
\end{lstlisting}

\begin{lstlisting}[language=json,firstnumber=1]
  {
    "type"          : "file_download", 
    "label"         : "<h3>Download computed optical flow:</h3>",
    "contents"      : { "tiff": "stuff_tvl1.tiff", 
                        "flo" : "stuff_tvl1.flo",
                        "uv"  : "stuff_tvl1.uv" }  },
\end{lstlisting}
Example using 'repeat':
\begin{lstlisting}[language=json,firstnumber=1]
  { "type"          : "file_download", 
    "repeat"        : "params.scales",
    "label"         : "'Download the estimations obtained at scale '+idx",
    "contents"      : "'estimation_s'+idx+'.txt'" }
\end{lstlisting}

%------ html_text  ------
\subsection{ \emph{html\_text} type}

\begin{longtable}{|>{\bf}L{\dimexpr 0.15\linewidth}|L{\dimexpr 0.7\linewidth}|c|}
\hline
\centering {key}     & \centering {\bf description} & {\bf req} \tabularnewline 
\hline \hline
 type      & html\_text  & yes \\ \hline
 visible   & Evaluated Javascript boolean expression 
             which hides the result if
             evaluated to false (if undefined the result is visible)& no \\ \hline
 contents  & array of strings, that will be concatenated to form the html 
             content. This content can contain Javascript expression if it starts
            with a single quote. & yes \\ \hline
\caption{Keys for the 'html\_text' type.}
\end{longtable}

\paragraph{Example}:\\
\begin{lstlisting}[language=json,firstnumber=1]
  { "type"          : "html_text", 
    "contents"      : [
      "'<p style=\"font-size:85%\">",
        "* &ldquo;Exact&rdquo; is computed with FIR, ",
        "DCT for &sigma;&nbsp;&gt;&nbsp;2 ",
        "(using '+params.sigma<=2?'FIR':'DCT'+",
      "'</p>'" ] },
\end{lstlisting}


%------------ text_file --------------------------
\subsection{ \emph{text\_file} type}

\begin{longtable}{|>{\bf}L{\dimexpr 0.15\linewidth}|L{\dimexpr 0.7\linewidth}|c|}
\hline
\centering {key}     & \centering {\bf description} & {\bf req} \tabularnewline 
\hline \hline
 type      & text\_file  & yes \\ \hline
 visible   & Evaluated Javascript boolean expression 
              which hides the result if
             evaluated to false (if undefined the result is visible)& no \\ \hline
 label     & HTML label & yes \\ \hline
 contents  & text filename to display & yes \\ \hline
 style     & associated CSS style & yes \\ \hline
\caption{Keys for the 'text\_file' type.}
\end{longtable}

\paragraph{Example}:\\
\begin{lstlisting}[language=json,firstnumber=1]
  { "type"          : "text_file", 
    "label"         : "<h2>Output<h2>",
    "contents"      : "stdout.txt",
    "style"         : "width:40em;height:16em;background-color:#eee" }
\end{lstlisting}

%------ warning  ------
\subsection{ \emph{warning} type}

\begin{longtable}{|>{\bf}L{\dimexpr 0.15\linewidth}|L{\dimexpr 
0.7\linewidth}|c|}
\hline
 \centering {key}     & \centering {\bf description} & {\bf req} 
\tabularnewline \hline \hline
 type      & warning  & yes \\ \hline
 visible   & Evaluated Javascript boolean expression 
            which hides the result if
             evaluated to false (if undefined the result is visible)& no \\ \hline
 contents  & displayed text, can contain both HTML tags and Javascript code.
expressions of the form '\{\{expression\}\}'. & \\ \hline
\caption{Keys for the 'warning' type.}
\end{longtable}
\paragraph{Example}:\\
\begin{lstlisting}[language=json,firstnumber=1]
{ "type":"warning", 
  "condition":"sizeX * sizeY < X",
  "contents":"'Needs X pixels ('+sizeX *sizeY+' given)<br/>'"},
\end{lstlisting}


%------ message  ------
\subsection{ \emph{message} type}
It displays a text message with a predefined color. This can be 
used for warning or error messages.

\begin{longtable}{|>{\bf}L{\dimexpr 0.15\linewidth}|L{\dimexpr 0.7\linewidth}|c|}
\hline
\centering {key}     & \centering {\bf description} & {\bf req} \tabularnewline 
\hline \hline
 type       & message  & yes \\ \hline
 visible    & Evaluated Javascript boolean expression which hides the result if
              evaluated to false (if undefined the result is visible)& no \\ \hline
 contents   & a string which will be evaluated by Javascript to get the message. & yes \\ \hline
 textColor    & the name of a color or a CSS-compatible color. & no \\ \hline
\caption{Keys for the 'message' type.}
\end{longtable}

\paragraph{Examples}:\\
\begin{lstlisting}[language=json,firstnumber=1]
  {    "contents": "'Image too small: the input image needs to be at least 42000 pixels to get a reliable estimate<br> Forced to use one bin for the estimation.'", 
        "type": "message", 
        "textColor": "red",
        "visible": "info.sizeX * info.sizeY < 42000" }
\end{lstlisting}
