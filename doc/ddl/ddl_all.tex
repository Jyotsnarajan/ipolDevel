\section{The Demo Description Language}

current version is 0.2.

\subsection{Introduction}
The Demo Description Language (DDL) is a textual description of a IPOL demo that 
allows to compile, generate a web interface, and run a demo. The description is 
written in JSON (JavaScript Object Notation) format, which is a standard format 
used in web applications. Current the JSON files are located in the 'static/JSON'
subdirectory. This language is evolving to allow a maximum number 
of demos to be described without the need to manually write HTML or Python code 
for a given demo. Each main key in the description file is described in the 
following sections:
\begin{itemize}
  \item \textit{general}: some general options;
  \item \textit{build}: everything needed to download and compile the source code;
  \item \textit{inputs}: description of the different inputs;
  \item \textit{params}: description of the parameters, for the param page;
  \item \textit{run}: commands to run the demo;
  \item \textit{config}: configuration, saving demo information;
  \item \textit{archive}: describes what is saved in archive when inputs are uploaded;
  \item \textit{results}: displaying the result page.
\end{itemize}



{\bf Note:} in JSON format, always use double quotes around keys or string 
values, not single quotes. Use can use the following link (or any other validator)
to check if your JSON code is valid: http://jsonlint.com/.
Also be carreful with commas, missing or additional commas are often the cause of
invalid JSON files.

%-------------------------------------------------------------------------------
\subsection{The \emph{general} section}
The general section describes global information about the demo.
It is a set of  (key,value) pairs, described in the following table.
Many keys are derived from the static variables of the previous 'app' Python class.\\
% 
\begin{longtable}{|>{\bf}L{\dimexpr 0.28\linewidth}|L{\dimexpr 
0.57\linewidth}|c|}
\hline
 \centering {key}     & \centering {\bf description} & {\bf req} 
\tabularnewline \hline \hline
 demo\_title         & demo title & yes\\ \hline
 input\_description  & description at the top of the input selection page, 
                      contains html as a single string or as an array of string
                      that will be concatenated and separated with spaces.
                     & yes \\ \hline
 param\_description  & description at the top of the para\-meters page,
                      contains html as a single string or as an array of string
                      that will be concatenated and separated with spaces.
                      & yes
                      \\ \hline
 xlink\_article     & defines the link to the article webpage & yes  \\ \hline
 inpainting         & boolean: if true, the demo is an inpainting demo with a 
                      single input image and mask drawing is enabled. & no \\ \hline
 crop\_maxsize      & limit allowed crop size in both width and height, the string
                      can contain a Javascript or JQuery expression which will be evaluated & no \\ \hline
 thumbnail\_size    & set initial thumbnail size for the input selection page & no \\ \hline
 input\_condition   & sets a condition on inputs using for example the input sizes,
                      it is written as an array of 3 string, the condition,
                      the exception sent, and the error message& no \\ \hline
 show\_results\_on\-\_error & show results even if the demo returned a run-time error.
                    the demo will be able to display pertinent information & no \\ \hline
\caption{Keys for the 'general' section ({\em req} means required).}
\end{longtable}

%-------------------------------------------------------------------------------
\subsection{The \emph{build} section}

The build section contains an array of build descriptions in the form
''[ \{...\}, \{...\}, ...] ''. For each description, an archive containing the 
source code is downloaded and compiled using either 'make' or 'cmake' features.
It has the following information:

\subsubsection{\emph{make} type}

\begin{longtable}{|>{\bf}L{\dimexpr 0.25\linewidth}|L{\dimexpr 0.6\linewidth}|c|}
\hline
\centering {key}     & \centering {\bf description} & {\bf req} \tabularnewline 
\hline \hline
 build\_type    & make & yes \\ \hline
 url        & full url link to download the demo source code & yes \\ \hline
 srcdir     & subdirectory from the extracted archive where the source code is 
            located & yes \\ \hline
 prepare\_make & shell command used to fix compilation errors of the source code,
                since the source code is steady & no  \\ \hline
 binaries   & list of binaries and their associated paths relative to the source 
            code path. In case of 'make' build type, the first path is the compilation
            path, where the make command is called, the second path is the binary
            path. You can run make without a binary name by setting the second
            path to a directory (it should end with '/' character), then all files 
            from this directory will be copied
            to the demo binary path. If the second path is a directory, give 
            a third value corresponding to one of the files to be copied, so that
            the system and check the file timetamp to decide if rebuild is needed.
            & yes \\ \hline
 flags      & make command compilation flags & yes \\ \hline
 scripts    & list of scripts and their associated paths relative to the source 
            code path & no  \\ \hline
 post\_build & shell command to run after the build & no \\ \hline
\caption{Keys for the 'make' type.}
\end{longtable}

\paragraph{Example}:\\
\begin{lstlisting}[language=json,firstnumber=1]
"build": [ 
  {
    "build_type"    : "make",
    "url"           : "http://www.ipol.im/../phs_3.tar.gz", 
    "srcdir"        : "phs_3",
    "binaries"      : [ [".","horn_schunck_pyramidal"] ],
    "flags"         : "-j4"
  {
    "build_type"    : "make",
    "url"           : "http://www.ipol.im/../imscript_dec2011.tar.gz", 
    "srcdir"        : "imscript",
    "binaries"      : [ [".","bin/", "plambda"] ],
    "flags"          : "-j CFLAGS=-O3 IIOFLAGS='-lpng -lm'"
  }
]
\end{lstlisting}

\subsubsection{\emph{cmake} type}

\begin{longtable}{|>{\bf}L{\dimexpr 0.25\linewidth}|L{\dimexpr 0.6\linewidth}|c|}
\hline
\centering {key}     & \centering {\bf description} & {\bf req} \tabularnewline 
\hline \hline
 build\_type  & cmake & yes \\ \hline
 url          & same as make type & yes \\ \hline
 srcdir       & same as make type & yes \\ \hline
 prepare\_cmake & shell command used to fix compilation errors of the source code,
                since the source code is steady & no  \\ \hline
 cmake\_flags  & cmake flags for configuration ('Release' build type is 
                automatically set) & no  \\ \hline
 binaries     & same as make type & yes \\ \hline
 flags        & same as make type & yes \\ \hline
 scripts      & same as make type & no  \\ \hline
 post\_build  & same as make type & no \\ \hline
\caption{Keys for the 'cmake' type.}
\end{longtable}

\paragraph{Example}:\\
\begin{lstlisting}[language=json,firstnumber=1]
"build": [ { 
  "build_type" : "cmake",
  "url"        : "http://www.ipol.im/xxx/ldm_q1p.zip", 
  "srcdir"     : ".",
  "binaries"   : [ [".","lens_distortion_correction"] ],
  "flags"      : "OMP=1 -j4" } ]
\end{lstlisting}

%-------------------------------------------------------------------------------
\subsection{The \emph{inputs} section}
The inputs section describes the set of input data of the algorithm (images, 
videos, flows, etc..).

\subsubsection{\emph{image} type}

\begin{longtable}{|>{\bf}L{\dimexpr 0.17\linewidth}|L{\dimexpr 0.68\linewidth}|c|}
\hline
\centering {key}     & \centering {\bf description} & {\bf req} \tabularnewline 
\hline \hline
 type         & image & yes \\ \hline
 description  & Short name or description of the input, it will used to display 
the input as the label inside the 'gallery' html display & yes \\ \hline
 max\_pixels   &  sets the maximal number of pixels of the input image, 
bigger images will be downsized, if 0 no resizing is done. The resizing as 
defined in the image class uses python PIL resizing with 'antialias' option. 
Input can be a number or a string that will be evaluated in Javascript.  & .. \\ \hline
 max\_weight   & max size (in bytes) of an input file, prevents uploading 
bigger files. Input can be a number or a string that will be evaluated in Javascript. & .. \\ \hline
 dtype        & input image expected data type, used as parameter of 
image.convert() method. Possible values are '1x8i' and '3x8i' which are 
converted respectively to 'L' and 'RGB' for PIL. & .. \\ \hline
 ext          & input image expected extention (ie. file format) & .. \\ \hline
\caption{Keys for the 'image' type.}
\end{longtable}

\subsubsection{\emph{flow} type}

\ToDo{[Karl]: explain how to assiocate an image to represent the flow. }


\begin{longtable}{|>{\bf}L{\dimexpr 0.17\linewidth}|L{\dimexpr 0.68\linewidth}|c|}
\hline
\centering {key}     & \centering {\bf description} & {\bf req} \tabularnewline 
\hline \hline
 type             & flow & yes \\ \hline
 description      & same as image & yes \\ \hline
 ext              & same as image & yes \\ \hline
 required         & boolean: tells if the flow is required for the algorithm
or optional, default is optional. For example, the ground truth may be optional.& no \\ \hline
\caption{Keys for the 'flow' type.}
\end{longtable}


%-------------------------------------------------------------------------------
\subsection{The \emph{params} section}
The params section describes the set of parameters needed by a demo, their 
constraints and their visual appearance. It is defined as an array of sets, 
where each set contains (key,value) pairs.


\subsubsection{ \emph{range} type}

The values of the range type are stored as numbers, so no double quotes are 
required around the values. The default value is stored with the 'values' field.

\begin{longtable}{|>{\bf}L{\dimexpr 0.15\linewidth}|L{\dimexpr 
0.7\linewidth}|c|}
\hline
 \centering {key}     & \centering {\bf description} & {\bf req} 
\tabularnewline \hline \hline
 type  & range       & yes \\ \hline
 visible  & Javascript/JQuery boolean expression evaluted in the context defined 
              in \ref{sec:JavascriptEvalParams} that can hide the parameter if
            evaluated to false (if undefined the parameter is visible)& no \\ \hline
 id     & parameter name in lowercase letters  & yes \\ \hline
 label  & name and/or description of the parameter, appears on the left side. & yes
                      \\ \hline
 comments & description of the parameter, appears on the right side. & no
                      \\ \hline
 values & set min,max,step and default values using the following key/value 
scheme \{ 'min':val, 'max':val, 'step':val, 'default':val \} & yes
                      \\ \hline
\caption{Keys for the 'range' type.}
\end{longtable}

\subsubsection{ \emph{range\_scientific} type}

The values of the range type are stored as numbers, so no double quotes are 
required around the values. The default value is stored with the 'values' field.
The 'range\_scientific' type differs from the 'range' type because it displays
the values in scientific notation with mantissa and exponent, the 'digits' parameter
sets the number of digits after the decimal points for the mantissa, the slider
will also to change each digit.

\begin{longtable}{|>{\bf}L{\dimexpr 0.15\linewidth}|L{\dimexpr 
0.7\linewidth}|c|}
\hline
 \centering {key}     & \centering {\bf description} & {\bf req} 
\tabularnewline \hline \hline
 type  & range\_scientific       & yes \\ \hline
 visible  & Javascript/JQuery boolean expression evaluted in the context defined 
              in \ref{sec:JavascriptEvalParams} that can hide the parameter if
            evaluated to false (if undefined the parameter is visible)& no \\ \hline
 id     & parameter name in lowercase letters  & yes \\ \hline
 label  & name and/or description of the parameter, appears on the left side. & yes
                      \\ \hline
 comments & description of the parameter, appears on the right side. & no
                      \\ \hline
 values & set min,max,step and default values using the following key/value 
scheme \{ 'min':val, 'max':val, 'digits':val, 'default':val \} & yes
                      \\ \hline
\caption{Keys for the 'range\_scientific' type.}
\end{longtable}


\subsubsection{ \emph{selection\_collapsed} type}

The values of the selection are stored as strings, so we map each label with a 
string value. The default value is stored in a separate field.

\begin{longtable}{|>{\bf}L{\dimexpr 0.25\linewidth}|L{\dimexpr 
0.6\linewidth}|c|}
\hline
 \centering {key}     & \centering {\bf description} & {\bf req} 
\tabularnewline \hline \hline
 type  & selection\_collapsed    & yes \\ \hline
 visible  & Javascript/JQuery boolean expression evaluted in the context defined 
              in \ref{sec:JavascriptEvalParams} that can hide the parameter if
            evaluated to false (if undefined the parameter is visible)& no \\ \hline
 id     & parameter name in lowercase letters & yes \\ \hline
 label  & name and/or description of the parameter, appears on the left side. & yes
                      \\ \hline
 comments & description of the parameter, appears on the right side. & no
                      \\ \hline
 values & set of (key,value) pairs, where the key is the displayed text and the 
value a string representing the corresponding value, for example \{ 
'black':'0', '1\%':'0.01' \} & yes
                      \\ \hline
 default\_value & defines the default value for this parameter, should be one 
the values defined in 'values'. & yes \\ \hline
\caption{Keys for the 'selection\_collapsed' type.}
\end{longtable}

\subsubsection{ \emph{selection\_radio} type}

The values of the selection are stored as strings, so we map each label with a 
string value. The default value is stored in a separate field.
This type is similar to the selection collapsed type, but display the different
options using radio buttons.

\begin{longtable}{|>{\bf}L{\dimexpr 0.25\linewidth}|L{\dimexpr 
0.6\linewidth}|c|}
\hline
 \centering {key}     & \centering {\bf description} & {\bf req} 
\tabularnewline \hline \hline
 type     & selection\_radio    & yes \\ \hline
 visible  & Javascript/JQuery boolean expression evaluted in the context defined 
              in \ref{sec:JavascriptEvalParams} that can hide the parameter if
            evaluated to false (if undefined the parameter is visible)& no \\ \hline
 id       & parameter name in lowercase letters & yes \\ \hline
 label  & name and/or description of the parameter, appears on the left side. & yes
                      \\ \hline
 comments & description of the parameter, appears on the right side. & no
                      \\ \hline
 vertical & boolean, if true use vertical display, otherwise use horizontal
            display (default=false) & no \\ \hline
 values   & set of (key,value) pairs, where the key is the displayed text and the 
value a string representing the corresponding value, for example \{ 
'black':'0', '1\%':'0.01' \} & yes
                      \\ \hline
 default\_value & defines the default value for this parameter, should be one 
the values defined in 'values'. & yes \\ \hline
\caption{Keys for the 'selection\_radio' type.}
\end{longtable}

\subsubsection{ \emph{label} type}

The label  can be used as a title to separate groups of parameters.

\begin{longtable}{|>{\bf}L{\dimexpr 0.27\linewidth}|L{\dimexpr 
0.58\linewidth}|c|}
\hline
 \centering {key}     & \centering {\bf description} & {\bf req} 
\tabularnewline \hline \hline
 type  & label       & yes \\ \hline
 visible  & Javascript/JQuery boolean expression evaluted in the context defined 
              in \ref{sec:JavascriptEvalParams} that can hide the parameter if
            evaluated to false (if undefined the parameter is visible)& no \\ \hline
 label & html text to display, as a single string or as an array of strings& yes
                      \\ \hline
\caption{Common keys for the 'label' type.}
\end{longtable}


\subsubsection{ \emph{readonly} type}

\begin{longtable}{|>{\bf}L{\dimexpr 0.27\linewidth}|L{\dimexpr 
0.58\linewidth}|c|}
\hline
 \centering {key}     & \centering {\bf description} & {\bf req} 
\tabularnewline \hline \hline
 type  & readonly    & yes \\ \hline
 visible  & Javascript/JQuery boolean expression evaluted in the context defined 
              in \ref{sec:JavascriptEvalParams} that can hide the parameter if
            evaluated to false (if undefined the parameter is visible)& no \\ \hline
 label & description of the parameter & yes \\ \hline
 id         & parameter name in lowercase letters  & yes \\ \hline
 value\_expr & Javascript/JQuery expression used to compute the parameter value, defined as an
              array of strings that will be concatenated to a single string.
            & yes \\ \hline
\caption{Common keys for the 'readonly' type.}
\end{longtable}

\subsubsection{ \emph{checkbox} type}

\begin{longtable}{|>{\bf}L{\dimexpr 0.27\linewidth}|L{\dimexpr 
0.58\linewidth}|c|}
\hline
 \centering {key}     & \centering {\bf description} & {\bf req} 
\tabularnewline \hline \hline
 type  & checkbox  & yes \\ \hline
 visible  & Javascript/JQuery boolean expression evaluted in the context defined 
              in \ref{sec:JavascriptEvalParams} that can hide the parameter if
            evaluated to false (if undefined the parameter is visible)& no \\ \hline
 label  & name and/or description of the parameter, appears on the left side. & yes
                      \\ \hline
 comments & description of the parameter, appears on the right side. & no
                      \\ \hline
 id         & parameter name in lowercase letters  & yes \\ \hline
 default\_value & boolean: True for checked & \\ \hline
\caption{Common keys for the 'checkbox' type.}
\end{longtable}

\subsubsection{ \emph{checkboxes} type}

\begin{longtable}{|>{\bf}L{\dimexpr 0.27\linewidth}|L{\dimexpr 
0.58\linewidth}|c|}
\hline
 \centering {key}     & \centering {\bf description} & {\bf req} 
\tabularnewline \hline \hline
 type  & checkboxes   & yes \\ \hline
 visible  & Javascript/JQuery boolean expression evaluted in the context defined 
              in \ref{sec:JavascriptEvalParams} that can hide the parameter if
            evaluated to false (if undefined the parameter is visible)& no \\ \hline
 label  & name and/or description of the parameter, appears on the left side. & yes
                      \\ \hline
 comments & description of the parameter, appears on the right side. & no
                      \\ \hline
 id         & parameter name in lowercase letters  & yes \\ \hline
 values     & list of dictionaries: [ {key:value, ...}, ... ]
            where key is the checkbox id and value is the text associated
            & yes \\ \hline
 default     & list of checkboxes that are checked by default  & yes \\ \hline
\caption{Common keys for the 'checkboxes' type.}
\end{longtable}


%-------------------------------------------------------------------------------
\subsection{The \emph{params\_layout} section}

The default layout of parameters is to stack them together within an 'HTML field'
entitle 'Parameters:'. However, if a demo needs to group together some parameters,
it can use a specific params\_layout section that contains an array of
sets of parameters, where each set is defined by an array containing the group 
title as the first element and the list of parameters ids as the second element,
where the parameters ids are their position in the params section (starting at 0).

\paragraph{Examples}:\\
\begin{lstlisting}[language=json,firstnumber=1]
  "params_layout": [
        [ "General parameters:",          [ 0,1]  ],
        [ "Sampled Gaussian kernel",      [2]     ],
        [ "Lindeberg's smoothing method", [3]     ]
    ],
\end{lstlisting}

%-------------------------------------------------------------------------------
\subsection{The \emph{run} section}

The 'run' section contains an array of elements, where each element can be 
a string or another array of strings:
\begin{itemize}
\item if the element is a string starting with '\#', it is considered as a comment.
\item if the element is a string, it can be a command to run from the system (shell, 
python script or demo binary) or a python code to evaluate if the string
starts with "python:",
\item if the element is an array of strings, the first string is a python
condition, if this condition is fulfilled, the remaining commands will be
executed.
\end{itemize}
In the command line, you can use \$param\_id to evaluate a 
python expression that can contain the parameter ids,
and '\textgreater output\_file' (without space between '\textgreater' and the 
filename) to redirect the standard output to a given file, save in the current 
working directory. 
If you also want to redirect the standard errors to the same file, use can add 
'2\textgreater\&1' as another argument. However, pipelines are not allowed 
('|' character).
Python scripts from the PythonTools directory can be used in commands: currently
'count\_lines.py' and 'image\_histogram.py'.


\paragraph{Examples}:\\
\begin{lstlisting}[language=json,firstnumber=1]
"run": [
  "#--- comment ---"
  "nlmeans  input_0.sel.png $sigma input_1.png output_1.png",
  "img_diff input_0.sel.png input_1.png $sigma output_2.png",
  "img_mse  input_0.sel.png output_1.png >stdout.txt 2>&1"
]
\end{lstlisting}

In this example, the last command adds a new parameters using self.algo\_params,
this parameter will be saved in the config files. It is also possible to add 
or change info and meta sections using self.algo\_info and self.algo\_meta.
\begin{lstlisting}[language=json,firstnumber=1]
  "run": [ 
    "mosaic     -p $pattern input_0.sel.png mosaiced.png ",
    "python:sizeX=x1-x0;sizeY=y1-y0",
    "python:zoom_factor=max(1, int(math.ceil(200.0/min(sizeX, sizeX))))",
    "python:sizeX=sizeX*zoom_factor;sizeY=sizeY*zoom_factor",
    "convert -filter point -resize ${sizeX}x${sizeY} input_0.sel.png    input_0.sel_zoom.png",
    "python:self.algo_params['zoom_factor']=zoom_factor"
  ]
\end{lstlisting}

A more complex example taken from demo 70:
\begin{lstlisting}[language=json,firstnumber=1]
  "run": [ 
      "convert input_0.sel.png inputNG.pgm",
      [ "threshold_type=='manual'", 
          "pgm2freeman -min_size $min_c -image inputNG.pgm -outputSDPAll -maxThreshold $tmax -minThreshold $tmin >inputPolygon.txt 2>algoLog.txt"
      ],
      [ "threshold_type=='auto'",   
          "pgm2freeman -min_size $min_c -image inputNG.pgm -outputSDPAll >inputPolygon.txt 2>algoLog.txt",
          "python:fInfo = open(self.work_dir+'algoLog.txt', 'r')",
          "python:lines = fInfo.readlines()",
          "python:line_cases = lines[0].replace(')', ' ').split();",
          "python:tmax = int(line_cases[17])"
      ],
      "python:contoursList = open (self.work_dir+'inputPolygon1.txt', 'w')",
      ...
  ]
\end{lstlisting}
 

%-------------------------------------------------------------------------------
\subsection{The \emph{config} section}

The config section is optional, it allows creating new (key,value) pairs in the 
configuration file based on text files obtained during the execution. This 
new information can then be used in the archive section.

\begin{longtable}{|>{\bf}L{\dimexpr 0.25\linewidth}|L{\dimexpr 0.6\linewidth}|c|}
\hline
\centering {key}     & \centering {\bf description} & {\bf req} \tabularnewline 
\hline \hline
 info\_from\_file    & it contains a list of (key,value) pairs where the key is
            the new information id to create (in lowercase letters) and the
            value is the corresponding text file that contains its contents & no \\ \hline
\caption{keys for the 'config' section.}
\end{longtable}

\paragraph{Example}:\\
\begin{lstlisting}[language=json,firstnumber=1]
"config":
  {
    "info_from_file": {  "homography_1" : "output_0.txt",
                          "homography_2" : "output_1.txt"
                      }
  }
\end{lstlisting}

%-------------------------------------------------------------------------------
\subsection{The \emph{archive} section}


\begin{longtable}{|>{\bf}L{\dimexpr 0.25\linewidth}|L{\dimexpr 0.6\linewidth}|c|}
\hline
\centering {key}     & \centering {\bf description} & {\bf req} \tabularnewline 
\hline \hline
 files    & (key,value) pairs where key is the file to archive and value is 
            the associated text information & no \\ \hline
 compressed\_\-files   & compressed files to add in the same format 
as files & no \\ \hline
 params  & list of parameters to archive & no \\ \hline
 info    & info variables to archive in the form of pairs variable:label & no \\ \hline
\caption{Keys for the 'text\_file' type.}
\end{longtable}


\paragraph{Example}:\\
\begin{lstlisting}[language=json,firstnumber=1]
"archive":
  {
    "files" : 
      { "input_0.png"                 : "input image",
        "primitives.txt"              : "Primitives"  },
    "params" :  
      [ "high_threshold_canny", 
        "initial_distortion_parameter", 
        "angle_point_orientation_max_difference" ],
    "info"   : { "run_time": "run time" }
  }
\end{lstlisting}

%-------------------------------------------------------------------------------
\subsection{The \emph{results} section}


The results section also contains an array of sets, where each set contains 
(key/value) pairs describing one type of output from the algorithm. There are 
displayed sequentially one below the other, a part from warnings that are 
displayed at the top of the page. The currently available types are described 
below. At the top of the results page, the processing time is displayed and the 
user is proposed to run the demo again with different input or parameters.


%------ gallery  ------
\subsubsection{ \emph{gallery} type}

The gallery type uses the gallery class to display images. Its parameters are:

\begin{longtable}{|>{\bf}L{\dimexpr 0.15\linewidth}|L{\dimexpr 0.7\linewidth}|c|}
\hline
 \centering {key}     & \centering {\bf description} & {\bf req} 
\tabularnewline \hline \hline
 type       & gallery  & yes \\ \hline
 visible    & Javascript/JQuery boolean expression evaluted in the context 
              defined in \ref{sec:JavascriptEval}, that can hide the result if
              evaluated to false (if undefined the result is visible)& no \\ \hline
 label      & html label for the gallery, can be either a single string or 
             a list of string that will be concatenated with whitespace character
              in between. & yes \\ \hline
 contents   & pairs of label:imagecontents information, where imagecontents can 
              have three different forms: a filename, an array of filenames, or
              a list of title:filename pairs. 
              If an image filename does succeed in loading,
              the image will be removed from the gallery. 
              You can also add a condition evaluated in Angular
              in the current context to enable or disable the pair display:
              using a label string of the form "condition?label".
            & yes \\ \hline
 options    & You can specify options for the ImageGallery object. Currently
              'minwidth' and 'minheight' (in pixels) options are available to force
              miminal size of image display. Default values are 300 for both.& no \\ \hline
\caption{Keys for the 'gallery' type.}
\end{longtable}

\paragraph{Examples}:\\
Standard example, using single filename strings.
\begin{lstlisting}[language=json,firstnumber=1]
  { "type"          : "gallery",
    "label"         : "<h2>Resulting images</h2>",
    "contents"      : { 
      "Input"     : "input_0.sel.png", 
      "Output"    : "output.png"
    },
    "options"     : { "minheight" : 400 }
  },
\end{lstlisting}
The use of '?' in the label "meta.hastruth?Ground truth" will condition the 
display of "t.ipol.png" image based on the boolean variable meta.hastruth.
\begin{lstlisting}[language=json,firstnumber=1]
  { "type"    : "gallery",
    "label"   : "<h2>Images</h2>",
    "contents": 
      {
        "Optical flow "              : "stuff_phs.ipol.png", 
        "meta.hastruth?Ground truth" : "t.ipol.png",
      }
   },
\end{lstlisting}
The use of pairs title:filename in the contents will display a HTML table
for each tab (here "DFT" and "DCT") with images and their associated titles.
\begin{lstlisting}[language=json,firstnumber=1]
  { "type"          : "gallery",
    "label"         : "<h2>Gaussian filter results</h2>",
    "contents"      : { 
      "DFT":
        { "1 iteration" :"dft_direct.png", 
          "N iterations":"dft_iter.png", 
          "difference"  :"dft_diff.png" }, 
      "DCT":                     
        { "1 iteration" :"dct_direct.png", 
          "N iterations":"dct_iter.png",
          "difference"  :"dct_diff.png" }
    }
  },
\end{lstlisting}
The use of string array in the contents will display HTML table without associated
titles.
\begin{lstlisting}[language=json,firstnumber=1]
  { "type"          : "gallery",
    "label"         : "<p>The following images were resized...<p>",
    "contents"      : { 
      "zoomed-out"  : [ "ipol_imgW.png", "palette1.png" ],
      "zoomed-in"   : [ "ipol_imgC.png", "palette1.png" ],
      "residue"     : [ "ipol_diff.png", "palette2.png" ],
      "mask"        : [ "ipol_mask.png", "palette1.png" ]
    }
  },
\end{lstlisting}


%------ file_download  ------
\subsubsection{ \emph{file\_download} type}

\begin{longtable}{|>{\bf}L{\dimexpr 0.15\linewidth}|L{\dimexpr 0.7\linewidth}|c|}
\hline
\centering {key}     & \centering {\bf description} & {\bf req} \tabularnewline 
\hline \hline
 type      & file\_download  & yes \\ \hline
 visible   & Javascript/JQuery boolean expression evaluted in the context defined 
              in \ref{sec:JavascriptEval}, that can hide the result if
             evaluated to false (if undefined the result is visible)& no \\ \hline
 repeat    & range expression (evaluated in Javascript):
              will create a loop in the form idx=0..range-1 & no \\ \hline
 label     & html title associated to the file download. In case of repeat, 
            evaluated as an expression with idx variable, otherwise, can be evaluated
            if it starts with single quote.& yes \\ \hline
 contents  & either a single string of the filename to download, or a list
              of label:filename pairs for files to download. In case of repeat, 
            evaluated as an expression with idx variable. & yes \\ \hline
\caption{Keys for the 'file\_download' type.}
\end{longtable}

\paragraph{Examples}:\\
\begin{lstlisting}[language=json,firstnumber=1]
  { "type"     : "file_download", 
    "label"    : "Download Hough result",
    "contents" : "output_hough.png" },
\end{lstlisting}

\begin{lstlisting}[language=json,firstnumber=1]
  {
    "type"          : "file_download", 
    "label"         : "<h3>Download computed optical flow:</h3>",
    "contents"      : { "tiff": "stuff_tvl1.tiff", 
                        "flo" : "stuff_tvl1.flo",
                        "uv"  : "stuff_tvl1.uv" }  },
\end{lstlisting}
Example using 'repeat':
\begin{lstlisting}[language=json,firstnumber=1]
  { "type"          : "file_download", 
    "repeat"        : "params.scales",
    "label"         : "'Download the estimations obtained at scale '+idx",
    "contents"      : "'estimation_s'+idx+'.txt'" }
\end{lstlisting}

%------ html_text  ------
\subsubsection{ \emph{html\_text} type}

\begin{longtable}{|>{\bf}L{\dimexpr 0.15\linewidth}|L{\dimexpr 0.7\linewidth}|c|}
\hline
\centering {key}     & \centering {\bf description} & {\bf req} \tabularnewline 
\hline \hline
 type      & html\_text  & yes \\ \hline
 visible   & Javascript/JQuery boolean expression evaluted in the context 
             defined in \ref{sec:JavascriptEval}, that can hide the result if
             evaluated to false (if undefined the result is visible)& no \\ \hline
 contents  & array of strings, that will be concatenated to form the html 
             content. This content can contain Javascript/JQuery expression if it starts
            with a single quote. & yes \\ \hline
\caption{Keys for the 'html\_text' type.}
\end{longtable}

\paragraph{Example}:\\
\begin{lstlisting}[language=json,firstnumber=1]
  { "type"          : "html_text", 
    "contents"      : [
      "'<p style=\"font-size:85%\">",
        "* &ldquo;Exact&rdquo; is computed with FIR, ",
        "DCT for &sigma;&nbsp;&gt;&nbsp;2 ",
        "(using '+params.sigma<=2?'FIR':'DCT'+",
      "'</p>'" ] },
\end{lstlisting}


%------------ text_file --------------------------
\subsubsection{ \emph{text\_file} type}

\begin{longtable}{|>{\bf}L{\dimexpr 0.15\linewidth}|L{\dimexpr 0.7\linewidth}|c|}
\hline
\centering {key}     & \centering {\bf description} & {\bf req} \tabularnewline 
\hline \hline
 type      & text\_file  & yes \\ \hline
 visible   & Javascript/JQuery boolean expression evaluted in the context 
              defined in \ref{sec:JavascriptEval}, that can hide the result if
             evaluated to false (if undefined the result is visible)& no \\ \hline
 label     & HTML label & yes \\ \hline
 contents  & text filename to display & yes \\ \hline
 style     & associated CSS style & yes \\ \hline
\caption{Keys for the 'text\_file' type.}
\end{longtable}

\paragraph{Example}:\\
\begin{lstlisting}[language=json,firstnumber=1]
  { "type"          : "text_file", 
    "label"         : "<h2>Output<h2>",
    "contents"      : "stdout.txt",
    "style"         : "width:40em;height:16em;background-color:#eee" }
\end{lstlisting}

%------ warning  ------
\subsubsection{ \emph{warning} type}

\begin{longtable}{|>{\bf}L{\dimexpr 0.15\linewidth}|L{\dimexpr 
0.7\linewidth}|c|}
\hline
 \centering {key}     & \centering {\bf description} & {\bf req} 
\tabularnewline \hline \hline
 type      & warning  & yes \\ \hline
 visible   & Javascript/JQuery boolean expression evaluted in the context 
            defined in \ref{sec:JavascriptEval}, that can hide the result if
             evaluated to false (if undefined the result is visible)& no \\ \hline
 contents  & displayed text, can contain both HTML tags and Angular
expressions of the form '\{\{expression\}\}'. & \\ \hline
\caption{Keys for the 'warning' type.}
\end{longtable}
\paragraph{Example}:\\
\begin{lstlisting}[language=json,firstnumber=1]
{ "type":"warning", 
  "condition":"sizeX * sizeY < X",
  "contents":"'Needs X pixels ('+sizeX *sizeY+' given)<br/>'"},
\end{lstlisting}


%------ repeat_gallery  ------
\subsubsection{ \emph{repeat\_gallery} type}
The repeat\_gallery type is design to display a gallery with a number of images 
that depend on a parameter, for example the number of scales used by the 
algorithm.

\begin{longtable}{|>{\bf}L{\dimexpr 0.15\linewidth}|L{\dimexpr 0.7\linewidth}|c|}
\hline
\centering {key}     & \centering {\bf description} & {\bf req} \tabularnewline 
\hline \hline
 type       & repeat\_gallery  & yes \\ \hline
 visible    & Javascript/JQuery boolean expression evaluted in the context 
              defined in \ref{sec:JavascriptEval}, that can hide the result if
              evaluated to false (if undefined the result is visible)& no \\ \hline
 label      & html label for the gallery, can be either a single string or 
             a list of string that will be concatenated with whitespace character
              in between.& yes \\ \hline
 repeat     & range expression (evaluated in Angular):
              will create a loop in the form idx=0..range-1 & yes \\ \hline
 contents   & an array where the first element is a string representing the label
              and that can contain Javascript/JQuery expression using the index idx, and
              the second element is either a string or an array of strings
              representing the image filenames to display. & yes \\ \hline
 options    & see 'gallery' type. & yes \\ \hline
\caption{Keys for the 'repeat\_gallery' type.}
\end{longtable}

\paragraph{Examples}:\\
\begin{lstlisting}[language=json,firstnumber=1]
  { "type"     : "repeat_gallery",
    "repeat"   : "params.scales",
    "label"    : "<h2> Noise Curves </h2>",
    "contents" : [ "'Scale '+idx+':'","'curve_'+idx+'.png'"],
    "options"  : { "minheight" : 600 },
\end{lstlisting}

\begin{lstlisting}[language=json,firstnumber=1]
  { "type"     : "repeat_gallery",
    "label"    : "<h2>Difference image (left), ...</h2> 
    "repeat"   : "params.nbscales",
    "contents" : ["'Scale'+idx",
                   [ "'denoised_diff_'+idx+'.png'",
                     "'denoised_noiseCurves_L_'+idx+'.png'",
                     "'denoised_noiseCurves_H_'+idx+'.png'"] ],
    "options"  : { "minwidth" : 400 }
\end{lstlisting}


%-------------------------------------------------------------------------------
\subsection{Full examples}

\paragraph{demo 104}: \\
\begin{lstlisting}[language=json,firstnumber=1]
{ 
  "general": { 
    "demo_title" : "An Analysis of the Viola-Jones Face Detection Algorithm",
    "input_description": "",
    "param_description": "",
    "enable_crop"      : true,
    "xlink_article": "http://www.ipol.im/pub/art/2014/104/"}, 
  "build": [ {
      "build_type"   : "make",
      "url"          : "http://www.ipol.im/pub/art/2014/104/vj_20140328.tar.gz", 
      "srcdir"       : "vj_20140328",
      "prepare_make" : "make OMP=1 -j -f makefile eigen",
      "binaries"     : [ [".","detect"] ],
      "flags"        : "OMP=1 -j -f makefile"
    } ],
  "inputs": [  {
          "type"            : "image", 
          "description"     : "input",
          "max_pixels"      : "900*900",
          "max_weight"      : "10*900*900",
          "dtype"           : "3x8i",
          "ext"             : ".png" } ],
  "params": [ 
    { "id"      : "layer_count",
      "type"    : "range", 
      "label"   : "Layer Count",
      "values"  : { "min":1,"max":31,"step":1,"default":31 }
    },
    { "id"      : "threshold",
      "type"    : "range",
      "label"   :  "Postprocessing Threshold",
      "values"  : { "min":0,"max":10,"step":0.1,"default":3 }
    } ],
  "run": ["detect input_0.sel.png $layer_count $threshold"
          "echo 'creating isGray information'",
          "python:im=image(self.work_dir+'input_0.sel.png')",
          "python:self.algo_info['isgray']=im.is_grayscale()"]
  "archive": {
      "files" : { 
          "input_0.sel.png"   : "exact image",
          "detectedraw.png"   : "raw detection",
          "ppSkin.png"        : "commands",
          "ppRobust.png"      : "skin color detection",
          "ppBoth.png"        : "most refined detection" } },
  "results": [
  {
    "type"          : "gallery",
    "label"         : "<h2>Resulting images</h2>",
    "contents"      : { 
      "Original"                        : "input_0.sel.png", 
      "Detected"                        : "detectedraw.png",
      "!(info.isgray)?Skin Check"       : "ppSkin.png",
      "Robustness Check"                : "ppRobust.png",
      "!(info.isgray)?Skin & Robustness Check":"ppBoth.png"
    }
  } ]
}
\end{lstlisting}

\subsubsection{keywords}

\section{Javascript/JQuery code}

\subsection{Evaluation of expressions in 'params' section} \label{sec:JavascriptEvalParams}

Javascript expressions used in the params section are evaluated in the 
following context of variables:
\begin{itemize}
  \item params object: contains the name and values of each parameter.
\end{itemize}

\subsection{Evaluation of expressions in 'results' section} \label{sec:JavascriptEval}

Javascript expressions used in the results section are evaluated in the 
following context of variables:
\begin{itemize}
  \item idx: index of the current loop if available,
  \item sizeX, sizeY: input image dimensions (after possible crop),
  \item info: the 'info' object defined in the config and run sections,
  \item meta: the 'meta' object,
  \item params: the 'params' object containing the values of the parameters,
  \item imwidth, imheight: maximal dimension of the input images,
  \item work\_url: url link to the outputs.
\end{itemize}

Some properties are automatically evaluated as expressions, like boolean expressions
and repeat expressions. Others like strings can be evaluated as Javascript expressions
if the string starts with a single quote.


% \ToDo{[Miguel] This section must dissapear (comment it out with begin-comment, end-comment.}
% 
% Angular is used as a flexible tool in combination to CherryPy to create the web 
% pages from the demo descriptions. Angular files are located in 'static/Angular'
% directory, while the template HTML files are located in 'lib/template' directory.
% The main Angular files are app.js, controllers.js and services.js located in the
% 'static/Angular/js' directory.
% The app.js file defines the Angular application, name "IPOLDemosApp".
% 
% \subsection{Services}
% Current services are Demo, Params, Meta and Info. They read and return the contents
% of a JSON file.
% 
% \subsection{Controlers}
% \subsubsection{DemoInputCtrl}
% The demo input controler defines the following variables in its scope:
% \begin{itemize}
%   \item demo\_id: taken from the global variable, using CherryPy
%   \item blob\_server: taken from the global variable, using CherryPy
%   \item demo: the demo information read from the JSON file using the Demo service
%         the demo is preprocessed by the method PreprocessDemo() defined in app.js
%         which can set default values and preprocess some fields.
%   \item demoblobs: read from the webservice 'get\_blobs\_of\_demo\_by\_name\_ws' on the 
%         blob\_server. Initially implemented as a service, it has changed to 
%         a \$http.get() command to allow the use of the blob\_server variable.
%         Once the information is obtained, the variable html\_params is created
%         that contains the HTML parameters associated to each blob selection.
%   \item a DisableBlobDisplay() method disables the display of a blob if it couldn't
%         be loaded.
% \end{itemize}
% 
% \subsubsection{DemoParamCtrl}
% 
% The demo input controler defines the following variables in its scope:
% \begin{itemize}
%   \item current\_scope
%   \item demo\_key and demo\_id given by Python in the HTLM template
%   \item Math (for calculation), maxdim, display\_ratio
%   \item CropInfo which contains the boolean enabled, the coord structure and
%         the minsize information.
%   \item the boolean variables got\_meta, got\_param, got\_demo that will be true
%         when the corresponding json files are loaded.
%   \item a function 'initParams' that initialize the parameters obtained from 
%         the DDL json file
%   \item a function updateCropInfo() that compute the real crop information
%         based on the displayed crop (since there is a zoom factor)
%   \item the initialization of demo, params, meta variables using the associated
%         services and calling the PreprocessDemo(),initParams() and updateCropInfo()
%         methods when needed, also updating display\_ratio, imwidth, imheight, 
%         got\_\{demo,param,meta\},
%         and crop\_image\_url variables.
%   \item additional functions: renderHtml, CheckString, CheckArray, 
%         Disable/LoadedImage that allow hiding images that don't load.
% \end{itemize}
% 
% \subsubsection{DemoWaitCtrl}
% 
% The controller is used to update a timer of the number of seconds that the algorithm
% takes to execute.
% 
% \subsubsection{DemoResultCtrl} \label{sec:DemoResultCtrl}
% 
% This controller contains the following variables in its scope:
% \begin{itemize}
%  \item the function initResults() that adds a load status to each image
%         in a gallery, initial status is 'trying'.
%  \item useful variables idx for loop, maxdim, current\_scope, Math, demo\_id,
%       work\_url
%   \item 3 display parameters display={param1,param2,param3} that can be used
%         by demos that need specific display parameters.
%   \item the demo,params,meta and info parameters read from json files.
%   \item the dimensions of the (possibly cropped) input image: sizeX and sizeY 
%   \item the maximal dimension of original input images: imwidth and imheight and 
%         the display ratio: display\_ratio.
%   \item a few more useful functions.
% \end{itemize}

% 
% \subsection{Templates}
% \subsection{Image crop component}
