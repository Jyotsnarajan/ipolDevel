\section{The Demo Description Language}

current version is 1.0

\subsection{Introduction}
The Demo Description Language (DDL) is a textual description of a IPOL demo that 
allows to compile, generate a web interface, and run a demo. The description is 
written in JSON (JavaScript Object Notation) format, which is a standard format 
used in web applications. This language is evolving to allow a maximum number 
of demos to be described without the need to manually write HTML or Python code 
for a given demo. Each main key in the description file is described in the 
following sections.


{\bf Note:} in JSON format, always use double quotes around keys or string 
values, not single quotes.

\subsection{The \emph{general} section}
The general section describes global information about the demo. It is a set of 
(key,value) pairs, described as follows.
\subsubsection{demo\_title}
The demo title.
\subsubsection{demo\_input\_description}
Description at the top of the input selection page.
\subsubsection{param\_title}
Description at the top of the parameters page.

\subsection{The \emph{params} section}
The params section describes the set of parameters needed by a demo, their 
constraints and their visual appearance. It is defined as an array of sets, 
where each set contains (key,value) pairs.

\subsubsection{ \emph{selection\_collapsed} type}
\subsubsection{ \emph{range} type}
\subsubsection{ \emph{label} type}

\subsection{The \emph{results} section}

\subsubsection{ \emph{run\_again} type}
\subsubsection{ \emph{warning} type}
\subsubsection{ \emph{image} type}
\subsubsection{ \emph{repeat\_image} type}
\subsubsection{ \emph{gallery} type}
\subsubsection{ \emph{repeat\_gallery} type}
\subsubsection{ \emph{html\_text} type}
\subsubsection{ \emph{file\_download} type}

\subsection{The \emph{build} section}
\subsubsection{url}
The full url link to download the demo source code.
\subsubsection{srcdir}
Subdirectory from the extracted archive where the source code is located.
\subsubsection{binaries}
List of binaries and their associated paths relative to the source code path.
\subsubsection{flags}
Cmake compilation flags.
\subsubsection{scripts}
List of scripts and their associated paths relative to the source code path.

\subsection{The \emph{run} section}
\subsubsection{keywords}

\subsection{Examples}


\subsubsection{keywords}
