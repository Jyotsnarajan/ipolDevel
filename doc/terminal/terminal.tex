\section{The Control Terminal}

\subsection{Introduction}
The Control Terminal is a small application intended to system administrators which allows to start, stop, and query the status of each module. It is composed of an xml file describing each modules, and of the terminal itself, a python script. \\
Upon launch, a prompt will appear, asking the user for a command.

\subsection{Architecture}
The terminal allows the user to type a variety of commands, for controlling the states of each modules composing the IPOL system. Some take parameters, generally the name of a modules. Every command is coded in the terminal script, and the xml file list for every module, the commands that can be used with this module as parameter.

\paragraph{XML file} \hspace{0pt} \\
The xml file ``modules.xml'' is parsed at the beginning of the execution of the python terminal for storing a dictionnary describing each module. \\
For security purposes, it is important to stress that the IPOL system must be deployed in a way of making this file trusted input : a call to the system() function in the terminal using content of this file open the way to a shell remote exploit if someone were to modify it. \\
Each module is defined by the attribute ``name'' and contain a variable number of tags : ``url'', ``server'', and ``path'', each describing respectively the url where the services provided by the module can be accessed, the server where the module is, and the path to the modules directory on said server. After that, an undefined number of ``command'' tags allow the module to be given as parameter for said commands. In order to function, the commands start, ping and shutdown.

\paragraph{Structure of the dictionnary parsed from the XML file} \hspace{0pt} \\
The dictionnary has, as keys, the name of each modules, and as value, another dictionnary with the following keys : ``url'', ``server'', ``path'', ``commands''. All of them but ``commands'' take as value the strings in the XML file server. The last key, ``commands'', takes as value a list of strings, each being a command available to the module. This list contains every string in ``command'' tags in the xml file, plus the string ``info'', which is added automatically. This dictionnary is the only attribute of the terminal object, initiated as ``self.dict\_modules''.

\paragraph{Execution loop} \hspace{0pt} \\

\subsection{Commands}
\paragraph{start} \hspace{0pt} \\
Usage :
\begin{verbatim}
start <module>
\end{verbatim}
\paragraph{ping} \hspace{0pt} \\
Usage :
\begin{verbatim}
ping <module>
\end{verbatim}
\paragraph{shutdown} \hspace{0pt} \\
Usage :
\begin{verbatim}
shutdown <module>
\end{verbatim}
\paragraph{info} \hspace{0pt} \\
Usage :
\begin{verbatim}
info <module>
\end{verbatim}
\paragraph{modules} \hspace{0pt} \\
Usage :
\begin{verbatim}
modules
\end{verbatim}
\paragraph{help} \hspace{0pt} \\
Usage :
\begin{verbatim}
help
\end{verbatim}
