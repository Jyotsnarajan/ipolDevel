% Nginx as a reverse proxy

\section{Nginx as a reverse proxy}
The IPOL project uses Nginx as a reverse proxy in order to take internet requests and send them to the servers in the 
internal network. When Nginx proxies a request, it sends the request to a specified proxied server, fetches the response, 
and sends it back to the client. With Nginx we implement private demos, microservide architechture patter and serve static 
files used by the clients.

\subsection{Static files}
The control panel as a Django web application needs to be compiled everytime the files change, so in order to serve the 
application Django offers a cli tool to collect all static files. Nginx serves all of the static for the application and exposes them 
under a pubic route.

\subsection{API routing}
Nginx is also used to redirect incoming requests to the corresponding port and direction inside the machine that has the 
module running. For each module runnning inside a server there is an entry in the sites-available file specifying where to 
route the request and deliver it to it's desired endpoint.

\subsection{Private demos}
In order to have private demos that require an authentication step with username and password the IPOL system makes 
use of Nginx. Every demo starting with id 33333001 will require authentication. This is controlled using routing logic. If this 
authentication fails Nginx is configured to forbid access to the demo automatically.