\section{Access control}
\label{sec:Access_control}
\subsection{Introduction}
To avoid the execution of exposed methods that allow the modification of the database (write and delete actions) by an unauthorized 
client, it has been created a wrapper that controls the access to these methods.

This wrapper compares the IP of the client that is calling the method with a list of authorized patterns. If the IP is authorized, 
the wrapper will automatically invoke the method requested by the client. Otherwise the request will be denied and the wrapper will 
respond with an Authentication Failed message.


\subsection{Authorized IPs File}
The file with the authorized IPs patterns is authorized\_pattern.conf, in the config\_common directory. This file consists
of a general section [Patterns], where all the permitted patterns are given with the following format: ``name=pattern''. Each pattern
allows the access of multiple IPs. 

The syntax for the pattern uses the asterisk to mean ``any''. For example: {\tt``local = 127.*.*.*''}

\subsection{Usage}
To add the authentication requisite to a method it is enough to add the {\tt@authenticate} decorator. It is important to add it 
below the {\tt@cherrypy.expose} decorator. Otherwise the method will not be longer exposed.
