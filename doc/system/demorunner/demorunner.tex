
\section{DemoRunner module}
\label{sec:DemoRunner}

This module controls the execution of the IPOL demos. It can execute directly its binaries or supporting scripts (provided by the demo editors) related to a particular demo (demoextras). Besides, a demo editor can use some generic scripts (PythonTools) that helps for representing the results of a demo such as draw 2D curves, draw histograms, counting lines and similar. This scripts are stored in each DemoRunner module. 

IPOL offers several machines for the demo executions due to its distributed system. Thanks to this, the Core can request to run the algorithms into the machine that best fits according to the policies of the Dispatcher module. For this, DemoRunner is responsible of informing the Core about the load of the machine where it is running. This allows to have several machines with different requirements for the demos (Matlab, specific libraries, etc.) and more concurrent executions due to more machines. 

Once the Core obtains the best machine for the current demo execution, the DemoRunner ensures that the experiment is done with the last source codes provided by the authors. This is achieved by downloading and compiling the source codes (stored in a compressed file) directly from an URL that must be given by the DDL.

The first time that a demo is executed, the IPOL system gets the codes from the URL, stores them in the DemoRunner machine and, then, it extracts and compiles them for the execution. If this process success, the module moves the requested executables to the corresponding directory of the demo and keeps them for the following executions. Henceforth, IPOL will check modifications in the http headers provided by the URL and the metadata of the codes stored. If there are any modifications in the content or the dates of the file stored in the DemoRunner, IPOL will download and compile the source codes again. 

The second intervention of the module is the execution itself. The Core provides all the information that the DemoRunner requires such as info to construct the running path in the shared folder, the DDL run section and the parameters set by the user. Next, the DemoRunner prepares the enviroment for the execution by replacing the variables with its values (the user parameters and demoextras, matlab and binaries paths) and return the command to be executed. 

Once the execution is finished, the module responses with the results. DemoRunner provides an interesting mechanism for recovering the info. The editor of a demo can store information from the execution in a text file (algo\_info.txt). DemoRunner recovers the information and give it to the core. Then, the core can return it to the web interface so it can be shown according to the DDL specifications.

The module takes care of stopping the demo execution if a problem appears such as not supported inputs for the demo, a bad implementation of the source codes or the demoextras, etc; then, the DemoRunner inform the Core about the causes of the failure so the Core can take the best action in response. Another issue is dealing with timeout that occurs when a demo exceeds a reasonable time for its execution, either because of problems in the code or simply because the execution time exceeds a reasonable time. This execution time can be indicated in the DDL of the demo offering the possibility that the editor decides what he considers reasonable time. Otherwise, the IPOL system will assign its own value and likewise, if the time set in the DDL is excessive or too short, the DemoRunner will modify this to a more reasonable value.