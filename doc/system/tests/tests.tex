\section{Tests}
\label{sec:Tests}
\subsection{Introduction}

The objective of the tests is to analyze automatically if the system fits the intended use. For that purpose the system is analized with two
types of tests, the first one are the Integration Tests in which individual software modules are combined and tested as a group, this implies from
the creation of a demo to its execution. The second one are the Unit Tests that are executed for each module and checks the response of all the
exposed methods.

Those tests are automatically executed in the integration and production enviroments every day by the crontab and every time a pull is made by 
the terminal.py script. If any of those test fail, an email is sent to all the emails listed in the /ci\_tests/send\_to.txt file.


\subsection{Integration Tests}

The integration tests are located in the /ci\_tests/system.py script and are responsible for testing all the IPOL modules and the interactions
between them. To accomplish that goal each test execute the full flow of the system, that goes from the creation of a demo, adding the DDL, blobs,
demoExtras, etc. to the execution of it.

Each test is independent from the others so the order in which it is executed is not important. To ensure that the state of the system has not been
altered by other test, the function setUp() is executed before every test. This function is responsible for cleaning the database from the 
remains of other tests.

\subsection{Unit Tests}

On each module there is a test.py script that executes all the unit test of the module. Each test check the correctness of every exposed method.

In the same way as the integration tests, each test can be executed in any order since the state of the system should not change after the execution
of the test.

\subsection{All.py}

This script is responsible for executing all the test, both the integration (system.py) and the unit test (test modules). To ensure
That several tests are not executed simultaneously, which could cause failures in the tests, a blocking system is implemented by the creation 
of the test.lock file, which if it exists implies that another instance of this script is running the tests.
If the script is blocked by this file it waits a random time between 5 and 10 seconds and check it again until it can run.


\subsection{Pull.sh}

This script is responsible for executing the git pull and call the all.py script to start all the tests. If any of the executed test fails an email
will be send to the email list reporting the failure.

This script is executed periodically every day at 10:00 am by the following instruction in the crontab: \begin{lstlisting}[language=Bash]
 0 10 * * * /home/ipol/ipolDevel/ci_tests/pull.sh
\end{lstlisting} in order to detect possible failures in the integration and production
enviroments, e.g. down modules, lack of disk memory, etc. Or every time a pull is made from the terminal.py script to check if the added cahnges do not
damage the system.
