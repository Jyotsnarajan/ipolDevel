% Project management and development methodology

\section{Project management and development methodology}
\miguel{For the moment this section is just a stub. Writing needed.}

\subsection{Project management}

\begin{itemize}
  \item We use CI
  \item Only commit when totally finished
  \item Trello
  \item Slack
\end{itemize}

\subsection{Development methodology}

\ToDo{Explain Continuous Integration, and the integration and production server}

Design Patterns \cite{GoF}, refactoring \cite{fowler1999refactoring}.

There are two git branches:

\begin{itemize}
  \item \textbf{master}: development
  \item \textbf{prod}: production
\end{itemize}

The {\tt master} branch is the default, where all development contributions are made. The testing server is configured to fetch this branch.

The {\tt prod} branch is for production. It is merged with {\tt master} only when the master is stable and one wants to integrate the changes in production. The production servers fetch this branch.

The {\tt prod} branch was created with:
\begin{verbatim}
git checkout -b prod
git push --set-upstream origin prod
\end{verbatim}

The .git/config ends up as:

\begin{verbatim}
[core]
	repositoryformatversion = 0
	filemode = true
	bare = false
	logallrefupdates = true
[remote "origin"]
	url = git@github.com:mcolom/ipolDevel.git
	fetch = +refs/heads/*:refs/remotes/origin/*
[branch "master"]
	remote = origin
	merge = refs/heads/master
[branch "prod"]
	remote = origin
	merge = refs/heads/prod
\end{verbatim}

To merge {\tt prod} with {\tt master:}
\begin{verbatim}
git checkout prod
git merge master
git push
git checkout master
\end{verbatim}

