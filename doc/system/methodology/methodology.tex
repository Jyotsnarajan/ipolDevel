% Project management and development methodology

\section{Development Methodology and Project Management}
The current IPOL project tries to follow the best practices in software engineering. Specifically, for this kind of project we found that Continuous Integration was a good choice in order to achieve fast delivery of results and ensuring quality. Continuous Integration is a methodology for software development proposed by Martin Fowler~\cite{fowler2006continuous} which consists on making automatic integrations of each increment achieved in a project as often as possible in order to detect failures as soon as possible. This integration includes the compilation and software testing of the entire project.

It is a set of policies that, together with continuous deployment, ensures that the code can be put to work quickly. It involves automatic testing in both integration and production environments. In this sense, each contribution in the IPOL system is quickly submitted and several automatic test are performed. If any of these tests fail the system sends an email indicating the causes.
%
Another advice of Continuous Integration is minimal branching. We use two. On one hand, master is the default branch and where all the contributions are committed. It is used for the development, testing and this continuous integration; on the other hand, the prod branch is used only in the production servers. It is merged with master regularly.
%
We use two different environments: integration and production. The integration server is where the master branch is pulled after each commit. The prod branch is used for the production servers and the code in this branch is assumed to be stable. However, the code in the integration server is also assumed to be stable and theoretically the code in the master branch could be promoted to production at any time once it has been deployed to the integration server and checked that is is fully functional and without errors. 

Quality is perhaps the most important requirement in the software guidelines of the IPOL development team. The code of the modules must be readable and the use of reusable solution is advised~\cite{GoF}. The modules must be simple, well tested and documented, with loose interface coupling, and with proper error logging. Note that it is not possible to ensure that any part of the IPOL will not fail, but in case of a failure we need to limit the propagation of the problem through the system and to end up with diagnostic information which allows to determine the causes afterwards.
%
Refactoring~\cite{fowler1999refactoring} is performed regularly and documentation is as important as the source code. In fact, any discrepancy between the source code and the documentation is considered as a bug.

Another tool used by the team is Trello. It allows to track the tasks of the project accoding to their current state (not assigned, assigned but not stated, assigned and in development, and finished). When a task arrives to the ``Finished" step, it is reviewed by the Project Director and archived (task totally finished) or moved back to development if more work is needed.

\begin{comment}
[ToDo]: move this to Sysadmin doc
The {\tt prod} branch was created with:
\begin{verbatim}
git checkout -b prod
git push --set-upstream origin prod
\end{verbatim}

The .git/config ends up as:

\begin{verbatim}
[core]
	repositoryformatversion = 0
	filemode = true
	bare = false
	logallrefupdates = true
[remote "origin"]
	url = git@github.com:mcolom/ipolDevel.git
	fetch = +refs/heads/*:refs/remotes/origin/*
[branch "master"]
	remote = origin
	merge = refs/heads/master
[branch "prod"]
	remote = origin
	merge = refs/heads/prod
\end{verbatim}

To merge {\tt prod} with {\tt master:}
\begin{verbatim}
git checkout prod
git merge master
git push
git checkout master
\end{verbatim}
\end{comment}
