\subsection{The Control Terminal}

The Control Terminal is an standalone application intended for system administration which allows to start, stop, and query the status of each of the IPOL modules. It reads the IPOL configuration from the XML files at {\tt ipolDevel/ipol\_demo/modules/config\_common}.

\subsubsection{Structure}
The Control Terminal is a command-line application to control the IPOL modules at each environment. The terminal is set to particular  enviroment to send the commands to the server in that environment. For example, the local, integration, or production environments. The current environment can be get or set with the {\tt env} command.

\paragraph{XML files} \hspace{0pt} \\
The XML files {\tt modules.xml} and {\tt demorunners.xml} are used by the IPOL modules when started. The Terminal reads these configuration files when started or when the environment is changed.

In {\tt modules.xml} the fields are:
\begin{itemize}
    \item module: the name of the module
    \item server: the host name of the server
    \item serverSSH: the name of the server (used to ssh it)
    \item path: the physical path of the module in the server
    \item command: it declares a command that the module is able to execute
\end{itemize}

The {\tt demorunners.xml} lists and configures each of the demoRunners in that enviroment.

\subsubsection{Commands}
\paragraph{start} \hspace{0pt} \\
Usage: {\tt start <module>}

It starts the specified module by ssh'ing to the server where the module is physically located and invoking the {\tt start.sh} script.
The {\tt ping} command might be executed right after to check if the module is indeed up.

\paragraph{ping} \hspace{0pt} \\
Usage: {\tt ping <module>}

It pings the module to check that it is responsive.

\paragraph{shutdown} \hspace{0pt} \\
Usage: {\tt shutdown <module>}

It shutdowns the specified module.

\paragraph{info} \hspace{0pt} \\
Usage: {\tt info <module>}

It prints the list of available commands for the specified module.

\paragraph{modules} \hspace{0pt} \\
Usage: {\tt modules}

It displays the list of the modules in the IPOL system.

\paragraph{env} \hspace{0pt} \\
Usage: {\tt env <environment>}

It prints the current enviroment when called without any parameters, or sets the specified enviroment.

\paragraph{help} \hspace{0pt} \\
Usage: {\tt help}

It prints the help of the Terminal.
