\section{The Core module}
The Core is the centralized controller of the whole IPOL demo system. It controls the execution of the experiments and delegates tasks such as data pre-processing (Conversion module), execution dispatching (Dispatcher module), algorithm execution (DemoRunner module), archiving experiments (Archive module), or retrieving demo metadata, among others. It also sends email notifications when failures are detected during the execution of a demo, bad constructed DDLs or any other problem that needs to be notified to the user, the technical staff, or the IPOL editors.

When an execution is requested, it first obtains its textual description (its DDL) from the DemoInfo module. Then, it asks for the workload of the different DemoRunners and gives this information to the Dispatcher module in order to pick the best DemoRunner according to the Dispatcher's policy. The next step is to ensure that the source codes are compiled and updated in the corresponding DemoRunner. If the demo uses a DemoExtras file it is updated with its last version.

For the execution, the Core creates a run folder, copies the input data, and delegates any eventual pre-processing to the Conversion module. The run folder is identified by an unique key and it is created inside a folder which can be accessed by any machine of the IPOL's architecture. Therefore, all the DemoRunner machines can access the shared folder where the executions are performed. The DemoExtras are common files that are also visible and centralized.

Once the execution folder is ready, the corresponding DemoRunner runs the algorithm with the parameters and inputs set by the user. The Core waits until the execution has finished or after a timeout. Finally, the Core ask the Archive module to store the experiment if the input data came from original data (uploaded by the user).

In case of any fatal failure (say, a conversion if needed but forbidden in the DDL, or the program of the article crashed), the Core terminates the execution and stores the errors in its log file. Eventually, it will send warning emails to the technical staff of IPOL (internal error) or to the demo editors.

Note that the Core does not need to know to which module it needs to delegate any operations, but instead simply requests the services using the IPOL API (see Sec. \ref{sec:reverse_proxy})

\subsection{DemoExtras}
\label{demoextras} 
For the execution of some demos it is necessary some support code or data, which we refer to as \emph{DemoExtras}. This is not part of the peer-reviewed or published material, and it is only used by the demo.

The support files are stored in a package (say, .tar.gz, .zip, .tar, ...) in the DemoInfo module. Also, a copy of this compressed file is stored in the ``dl\_extras'' folder in the ``shared\_folder'' for comparison reasons. \ToDo{This mechanism should be explained in the demoInfo module section of the doc.}

The first time a demo is executed, the demos extras are decompressed in the ``DemoExtras'' folder in the ``shared\_folder''. At each execution the Core checks the date and the size of the compressed file in the ``shared\_folder'' with the one stored in demoInfo. \ToDo{This mechanism should be explained in the demoInfo module section of the doc. A diagram would be useful.}

The possible results from the check are:
\begin{itemize}
 \item \textbf{Date and size match:} Nothing is done
 \item \textbf{Date or size do not match:} The DemoExtras are downloaded again
 \item \textbf{DemoExtras deleted in demoInfo:} All DemoExtras files related to that demo are deleted in the ``share\_folder''
\end{itemize}

\subsection{Editor-controlled demo failures}
The Editor has the possibility to detect that some conditions are not met and notify the system that the execution of the demo failed, even if the program did not crash or the running script exited with a sucessful exit code. Indeed, a demoExtras script in the demo could check, for example, if the aspect ratio of the image is what the algorithm excepts, and prevent the actual execution.

The mechanism is simple: the editors can write a "demo\_failure.txt" file and stop the execution with an exit
code 0. The Core interprets the presence of this file as the demo itself signalling a problem.

For example, this code is used in one of the demos:
\paragraph{Example}:\\
\begin{verbatim}
imageSize=$(identify -format '%w+%h' input_0.png)
maskSize=$(identify -format '%w+%h' mask.png)

if [ $imageSize != $maskSize ]
then
  echo "Input error: input and mask have different sizes." >> demo_failure.txt
  exit 0
fi
\end{verbatim} 
