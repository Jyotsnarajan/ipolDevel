\section{The Conversion module}
\label{sec:conversion}


\subsection{Introduction}
\label{sec:archive_introduction}

%\paragraph{Introduction} \hspace{0pt} \\
The Conversion module is a standalone application dedicated to conversions of input files (image and video). It is called by the Core module, to prepare the submitted files for an experiment, in their execution folder. Properties of each file are compared to the format expected by the demo, according to its DDL definition. The conversion module applies automatic transformations when it is possible. The dedicated code for image and video is isolated in a Python package, ipolutils. This package is shared by the Blobs and the Core module, for the creation of thumbnails.

\paragraph{Technologies} \hspace{0pt} \\
\begin{itemize}
\item Python (with packages available from PIP)
\item cherrypy (framework for webservices)
\item simplejson (required by cherrypy, JSON is the syntax to communicate with the module)
\item opencv-python (Python binding for the image processing library OpenCV, used for images and video)
\item tifffile (a Python binding for the libtiff library, used to load and save images in tiff format)
\item numpy (a Python library for matrix calculation, used to modify image data)
\end{itemize}


\subsection{Image library}

Multiple libraries are available in Python to load and save images. IPOL has some requirements, in order of importance:

\begin{itemize}
\item open source,
\item available as a pip package,
\item support of data type other than integer 8 bits, especially in tiff (uint16, uint32, float16, float32),
\item access to image data as a numpy matrix,
\item maturity and sustainability of the project,
\item intuitive API and good documentation,
\item fast.
\end{itemize}

\paragraph{OpenCV} \hspace{0pt} \\
Many libraries have been tested, the final choice has been OpenCV. OpenCV (Open Computer Vision) is a well known open source library in the domain of computer vision. It is a C++ program with a Python binding. After tests, it appears that loading and saving tiff files was not correct for uint32, float16 and float32 formats. The documentation could be improved, the API is a bit complex, but globally, OpenCV is efficient, with interesting algorithms (even they are not yet used by IPOL), and video support.

\paragraph{tifffile} \hspace{0pt} \\
Tifffile is a Python wrapper for the C libtiff library providing a numpy access to image data. This library seems actually the more robust, no bug has been found yet.

\paragraph{libtiff} \hspace{0pt} \\
Libtiff is a Python wrapper for the C libtiff library, providing numpy.memmap view of tiff data. This library is quite robust but a bug has been observed on reading images with endianess equals to ``MSB''.

\paragraph{pillow} \hspace{0pt} \\
Pillow is the most common Python library for images. It is well documented with a simple API, but, it does not support other formats than the common integer 8 bits.

\paragraph{ImageMagick} \hspace{0pt} \\
ImageMagick is a classical C library used with command line, and also available with other languages. There are too many Python bindings, most are badly supported or discontinued. Wand is the most known library, but it is not yet available by pip. There are also problems in ImageMagick API to control the Tiff parameters for saving.

\paragraph{GDAL} \hspace{0pt} \\
GDAL (Geospatial Data Abstraction Library) is a C library dedicated to geospatial images, with a Python wrapper. It handles a vast list of specific formats, and among them, have a decent tiff support. It is supposed to be available by pip, but installation does not work. It is possible to install python-gdal as a Debian package, but this produces inconsistency in Python packages when used in conjunction with pip, especially for numpy.

\paragraph{SimpleITK} \hspace{0pt} \\
ITK is a C++ library to process medical images. SimpleITK is a layer on ITK, supposed to simplify the API for script languages. The library is working well, available by pip, but the API is not stable enough to invest code on it.

\paragraph{pyPNG} \hspace{0pt} \\
PyPNG is a pure Python library, handling the PNG format. It could be very slow (40 times more than OpenCV to read big images). 

\paragraph{imageio} \hspace{0pt} \\
Imageio is a Python library that provides an interface to read and write a wide range of image data for numpy. But it has no original code, it silently choses among plugins like Pillow, OpenCV or SimpleITK, according to the extension, but without access to all the options of the original libraries.

\paragraph{scikit-image} \hspace{0pt} \\
Scikit is a Python library for image processing, designed to interoperate with scipy and numpy. It uses OpenCV behind the API.



